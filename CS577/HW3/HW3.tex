\documentclass[]{article}
\usepackage[letterpaper, margin=1in]{geometry}
\usepackage{amssymb}
\usepackage{amsmath}
\usepackage{graphicx}
\usepackage{algorithm}
\usepackage{algorithmicx}
%\usepackage{algorithmc}
\usepackage[noend]{algpseudocode}

\def\BState{\State\hskip-\ALG@thistlm}

\topmargin = -60pt

\title{HW 3}
\author{Yunhao Lin}

\begin{document}
\maketitle

\section{a:}

\begin{algorithm}[H]
    %\caption{My algorithm}\label{euclid}
   \begin{algorithmic}[1]
   \Procedure{NumShortestPath}{}

    \For {each vertecx V}
        \State initialize V's visited mark = 0
        \State initialize V's value = 0
        \State initialize V's count = 0
    \EndFor
    \\
    \State $\textit{q} = \text{New Queue()}$
    \State Mark \textit{v} as \textit{visited = 1}
    \State Mark the $\textit{v.value = 0}$ for length from the begin vertex to \textit{v}
    \State Mark the $\textit{v.count =1}$ for the number of possible ways to get here
    \State Add \textit{v} to \textit{q}
    \\
    \While {\textit{q} is not empty}
        \State mark \textit{z} as the next vertex in the \textit{q} (\textit{dequeue})
        \For{each successor \textit{c} of \textit{z}}
            \If {\textit{c} is \textit{unvisited}, \textit{visited = 0}}
                \State $c.value = z.value +1$
                \State $c.count = z.count$
                \State Mark $c$ as visited, $visited =1$
                \State Add $c$ to $q$
            \Else
                \If {$c.value == z.value + 1$}
                %\State  stay unchanged
                    \State $c.value$ \textbf{ stay unchanged}
                    \State $c.count += z.count$ 
                \ElsIf {$c.value > z.value + 1$}
                    \State $c.value = w.value +1$
                    \State $count = z.count$
                \ElsIf {$c. value < z.value +1$}
                    \State \textit{do nothing;}
                \EndIf
            \EndIf
            
        \EndFor
    \EndWhile
    \State Return the count value in $w$.
    \EndProcedure
    \end{algorithmic}
\end{algorithm}

\section{b:}
\textbf{Claim: } 
After every time a vertex is dequeued from the queue, which means
at each iteration of the while loop, the count is correctly set to every visited
vertex.\\
\textbf{Base Case:} 
At the begining vertex \textit{s}, each successor of \textit{s}
has its count set to 1 by the algorithm as they are added to the queue. Which is
correct and therefore base case hold. \\
\textbf{Inductive hypothesis:} 
After $k$ times of the iteration of the while loop,
each vertex $x$ that has been inspected 
holds the correct count of the number of shortest path to $x$.\\
\textbf{Inductive step:} 
Prove it worked for $k+1$ iteration, when the $(k+1)$th vertex is dequeued and 
update all the successor of $(k+1)$ vertex.\\
Name the (k+1)th vertex as $\alpha$. Every one of the neighbours of $\alpha$
has three conditions.\\
\textbf{Case 1:} 
The neighbour is never visited before, we name this neighbour $\beta$. Since the value of 
$\alpha$ is already the shortest from $v$ to $\alpha$, in this case, the value
of $(\alpha +1)$ would the shortest length from $v$ to $\beta$. And
the number of possible ways to get from $\alpha$ to $\beta$, is the 
same as the possible from $v$ to $\alpha$, cause there is only one way to get from
$\alpha$ to $\beta$. This indicates if $\alpha$ is never visited before, algorithm
would give the correct output for the number of shortest path at each vertex.\\
\textbf{Case 2:}
The neighbour is visited before we name it $\gamma$. So it already has a value and a count.
\begin{enumerate}
    \item When the path length from $v$ to $\alpha$ to $\gamma$, which is $\alpha +1 $ 
    is shorter than the path length already in $\gamma$.
    Update the shortest length in $\gamma$. And the number of possible ways to 
    get from $\alpha$ to $\gamma$, is the same as the possible from $v$ to 
    $\alpha$, cause there is only one way to get from $\alpha$ to $\gamma$.
    
    \item When the path length from $v$ to $\alpha$ to $\gamma$ is the same as the 
    the path length that is already in $\gamma$. The value $\gamma$ holds stay the 
    same, since it represents the length from $v$ to $\gamma$. But there is now 
    also path from $\alpha$ to $\gamma$ which is also the shortest path, the number 
    of possible shortest path is now the number of possible path to $\alpha$ plus
    the original possible path. Which our algorithm correctly states from line 
    23 to 24.This would correctly change $\gamma$'s value and count to represent the
    shortest path length and number of shortest path to $\gamma$.
    
    \item Whent he path length from $v$ to $\alpha$ to $\gamma$ is longer than the 
    path already in $\gamma$. There is no need to update the shortest path length
    and shortest path count in $\gamma$. Which is right.\\
    \\So for the neighbour that has been visited before, algorithm would create 
    the correct output.
\end{enumerate}

In general, in (k+1)th iteration, all successors of the (k+1)th vertex is updated
correctly. Therefore, the algorithm works for (k+1)th iteration. By induction,
The algorithm holds for number of n vertexs, n is a positive integer.

\section{c:}
Bascially, we use the implanmentation of BFS.
At first we actually iterate through list to set up the value and count which 
means there would be at most (m+n) time here too.
The operation of queuing and dequeue takes O(1) time, and we did that for
each vertex just once, which would give us the O(n). Since adjecency list
list only checked when the parent is dequeued, for each list there is only
one time that they can be checked. But we have three extra if statement here
which is O(1) and still makes the list checking in O(m) time. 
 In total, the time is still O(m+n). 
    


\end{document}