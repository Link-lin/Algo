\documentclass[]{article}
\usepackage[letterpaper, margin=1in]{geometry}
\usepackage{amssymb}
\usepackage{amsmath}
\usepackage{graphicx}
\usepackage{algorithm}
\usepackage{algorithmicx}
\usepackage{enumitem}
%\usepackage[linesnumbered,ruled,vlined]{algorithm2e} 
%\usepackage{algorithmc}
\usepackage[noend]{algpseudocode}

\def\BState{\State\hskip-\ALG@thistlm}

\topmargin = -60pt

\title{HW 4}
\author{Yunhao Lin}

\begin{document}
\maketitle

\begin{enumerate}
    \item In order to prove $S$ is feasible if and only if $\forall t \leq n$,
    the number of deliveries in $S$ due within $t$ days is no more than $t$. We
    need to prove that if $S$ is feasible, then for $\rightarrow \forall t \leq n$ 
    the number of deliveries in $S$ due within $t$ dyas is no more than t. And 
    also for $\forall t \leq n$ the number of deliveries in $S$ due with in $t$ days
    is no more than $t$ $\rightarrow S$ is feasible.
    \begin{itemize}
        \item We will use induction to prove that for every $t \leq n$, the 
        the number of deliveries in S due within t days is no more than t,
        then S is feasible.\\
        \textbf{Claim:} For every $t$ is smaller or equal to $n$,  the 
        number of deliveries due within t days is no more than t, then $S$
        is feasible.\\
        \textbf{Base case:} Zero customer is not meaningful to the process, 
        therefore we choose 1 as base case. There is only one customer which means,
        $n =1$.  Since we know that only one package can be delivered in one day. 
        So the number of package that can be delivered in one day is no bigger 
        than 1, which means the there is a way to deliver the one package in time.
        Our base case holds. \\
        \textbf{Inductive Hypothesis:} $P(n):$Assume for $\forall t \leq n$, the number
        of number of deliveries in $S$ due within $t$ days is no more than $t$,
        then S is feasible.\\
        \textbf{Induction Step:} We need to prove for $\forall t \leq n+1$,
        the number of deliveries in $S$ due within $t$ days is no more than $t$,
        then S is feasible. Since P(n) works, then the number of deliveries in $S$
        due within $t$ days is no more than $t$, and we can deliver at most $t$
        packages, for n +1, we have one extra package to deliver. And that package
        can be delivered on the extra day, since every t smaller than n+1, then the
        compared with P(n) there is one extra day to deliver the n+1 package. So,
        there is still a way to manage deliver all the package, which mean P(n+1)
        holds and S is feasible.\\\\
        Therefore, $P(n) \rightarrow P(n+1)$. By induction, for every t smaller 
        or equal to n, the number of deliveries is no more than t, then S is feasible.
        \end{itemize}

    \begin{itemize}
        \item If S is feasible, by definition, there is a way to order the deliveries
        in $S$ so that each one is made on time. Assume $S$ has $n$ packages, as 
        we know that only one package can be delivered in each day. So the least
        time to deliver $n$ packages would be $n$ days. Let's assume $\exists t 
        > n$, then the number of deliveries in $S$ due within $t$ days is no 
        more than $t$. So every day untill $t$ there can be a package delivered,
        so in total the maximum packages delivered could be $t$ packages. And
        by assumption, $t>n$, which contradicts the statement that deliveries
        in $S$ due within $t$ days cannot be more than $t$. By contradiction,
        $\forall t \leq n$, the number of deliveries in $S$ due within $t$ days is no 
        more than $t$. So no matter what $n$ number it is for $t \leq n$ , there is 
        always a way to deliver as more as $n$ number of packages, which means each
        package is delivered and it's feasible.f $S$ is feasible, then for 
        $\rightarrow \forall t \leq n$ the number of deliveries in $S$ due within
         $t$ dyas is no more than t. 


    \end{itemize}

    \item In order to maximize the profit. First we sort the due date $t_i$ in descending
    order. Therefore at 0 index is going to be the last due date. We then traverse
    through the due date in order from last to first. At each iteration, add
    all the $p_i$ corresponding to the day of the largest due date. Dequeue
    from the priority queue to get the highest profit available at that day. 
    By doing this, we guarantee that each day we choose is going to give us 
    the largest profit untill we reach the end of the package list.


    %\vspace{-\baselineskip}
    \begin{minipage}{\linewidth}
    \begin{algorithm}[H]
      \caption{Find maximum}
      \begin{algorithmic}[1]
        \Procedure{Maxprofit}{}
          \State Initialize $q$ = priority queue
          \State Quick Sort the List of $t_i$ in descending order to be $L$
          \State $k \gets $ largest due date
          \While{ k is not equal to 0}  
            \State Create a list of all packages due at $k$ \Comment{Get the due date from $L$}
            \State Add every $p_i$ at $k_{th}$ day to $q$
            \State Get highest priority $\beta$ from $q$ to gain highest profit on that day
             \\\Comment{if $q$ is empty, there is no delivery}
            \State  Mark dilivery of the package corresponding to $\beta$ at $k_{th}$ day
            \State $k \gets k-1$  \Comment{Update to the next date in $L$}
          \EndWhile
          
        \EndProcedure
      \end{algorithmic}
    \end{algorithm}
    \end{minipage}\\
    
    \textbf{Proof Of Correctness:}\\
    We will use induction to prove that algorithm would give us the maximized profit.\\
    \textbf{Claim:} The algorithm above would arrange deliveries delivers the largest
    profitable package at each day.\\
    \textbf{Base case:} Our base case is when $k$ is maximum $k$, the largest due date is $k$.
    Therefore a list of package is created at line 6 and add to $q$, the priority
    queue is added with all the package that has a due date of $k$. When we get highest
    priority from $q$, the largest number is returned so that is the most profitable
    job we need to deliver at $k$ days. Then we just find the corresponding package to
    diliver. Therefore at $k_th$ day we deliver the most profitable package. Our base case holds.\\
    \textbf{Inductive Hypothesis:} Algorithem would produce the most profitable 
    delivery at $k = x$, where x is less than the maximum of possible k. \\
    \textbf{Inductive Step:} We need to prove that algorithm would get the correct
    delivery at day $k = x-1$. After $k = x$ days and the package delivered.
    There is two possible situations at $k = x -1$:
    \begin{itemize}
      \item First, the queue after the $k = x$ delivery is not empty. Then we first
      add all the package price due at $k = x-1$ to the queue, therefore the queue
      contains everything that is due after $x-1$, which means these are all valid 
      package to deliver. Then algorithm find the maximum price we can do at $x-1$,
      then at $x-1_{th}$ day, the package that can get us most profit is marked to
      be dilivered at this date.
      \item Second, the queue after the $k = x$ is empty. Then as above, algorithm
      would find the package that can give us the greatest profit at $k = (x-1)$ day.
      Then the correct package would be marked to b dilivered at $k =(x-1)$ day.
    \end{itemize}
    Therefore, the P(k) holds then P(k-1) holds. The algorithm would end when the
    date is 0, by induction, every day before $k$ would deliver the most profitable
    package at the day. Therefore in total, that is the maximum profit we can get.
    And each package to be delivered is marked at the correct day to be delivered.
    \\
    The algorithm's complexity would be $O(klog_k)$, since we are doing k times
    every time the complexity is $log_k$. Our algorithm would detemine the correct
    order to deliver package to gain max profit.
      
    
  \end{enumerate}

\end{document}