\documentclass[14pt]{article}
\usepackage[letterpaper, margin=1in]{geometry}
\usepackage{amssymb}
\usepackage{amsmath}
\usepackage{graphicx}
\usepackage{algorithm}
\usepackage{algorithmicx}
\usepackage{enumitem}
\usepackage[noend]{algpseudocode}
    
\def\BState{\State\hskip-\ALG@thistlm}
    
\topmargin = -50pt
    
\title{HW 5}
\author{Yunhao Lin}
    
\begin{document}
\maketitle

\begin{enumerate}
    \item
    \textbf{Algorithm:}\\
    Since we are arranging books on m shelves. Know for j $\in$
    $\left\{ 1,...,m \right\}$, and $x_j = W- \sum_{i: book\, i\, is\, on\, shelf\,
    j}wi$, where $W$ is the width of each shelf. In order to minimize 
    $\sum_{j \in [m]}x_j$, the total waste space across $m$ shelves, we are arranging
    books in order. Assuming the next book to be put into $j_{th}$ shelf is $z_i$.
    While index of $i$ is less than the total number of books $n$, we 
    need to check if $x_j - w_i$ is greater than 0. If it is, then book $z_i$ is 
    able to put in $j_{th}$ shelf and we just put it there. If not, $z_i$ would
    be the first book in the next shelf.\\
    \textbf{Proof of Correctness:}\\
    \textbf{Claim:} The algorithm is going to help us minimize the total waste space
    , which is $\sum_{j \in [m]}x_j$.\\
    \textbf{Base Case:} Since the first book is always going to add at the first 
    book shelf, else we are wasting all the space in first shelf. It does not make
    sence to put it in the second shelf. \\So in this specific case , base case is 
    when have 2 books to put in shelf. In this scenario, the $1st$ book is already
    in the $1st$ shelf. And the algorithm determine if the $2nd$ can be put in the
    current shelf. If it can, then both book are in the same shelf, the total waste
    space would be ($W - w_1 - w_2$). If we did not put the book in $1st$ shelf,
    even though it can be put there, then $2nd$ book is put on the $2nd$ shelf,
    and the total waste is ($2W-w_1-w_2$) which is much larger then the algorithm 
    proposed. If $2nd$ book is too wide to put in $1st$ shelf, then it can only be
    put in the $2nd$ shelf as suggested by the algorithm.\\
    \textbf{Inductive Hypothesis:} P(n):Suppose when we have $k$ number of books, and
    the algorithm correctly provide with the minimized wasted space.\\
    \textbf{Inductive Steps:} Since we already all for $k$ books, algorithm
    would arrange it to minimize the waste space. We need to prove that for $(k+1)_{th}$
     book, the algorithm would still minimize the waste space. After $k_{th}$ book
    is added to the shelf, we are at $l_{th}$ shelf. And there is no need to worry 
    about any shelf before $l_{th}$ shelf, since every shelf at this moment is the
    minimized in waste space. So right now, in order to add in the $(k+1)_{th}$
    book, there is two possibilities.
    \begin{itemize}
        \item First, $(k+1)_{th}$ book can fit on $l_{th}$ shelf, then the algorithm add
        the $(k+1)_{th}$ book on $l_{th}$ shelf. And the totoal waste would be 
        all the waste for $k$ books $- w_i$. If we put this on the shelf below $l_{th}$ even though
        we can fit it in $l_{th}$ shelf. The total waste would be all the waste for
        $k$ books $W - w_i$, which means we are wasting a whole length of $W$. 
        And the algorithm provided with the best arrangement of the book.
        \item Second, $(k+1)_{th}$ book cannot fit on $l_{th}$ shelf, $x_l - w_{k+1}$
        is smaller than 0. Then $k+1$ book can only be put on the $(l+1)_{th}$ 
        shelf. The algorithm also provide the correct output.
    \end{itemize}
    Therefore, P(k) holds then P(k+1) holds. By induction, the algorithm would 
    correctly arrange books to minimize the total waste space, which is 
    $\sum_{j \in [m]}x_j$.  \\\\
    \textbf{Running time:} Since there is a total number of $n$ books, we 
    are doing comparsion and adding for each one of them. The total running 
    time could be analysed to be $O(n)$.\\\\\\\\\\
    \item Here is the psudo-code for No.2\\
    \begin{minipage}{\linewidth}
        \begin{algorithm}[H]
          \caption{Optimized book arrangement}
          \begin{algorithmic}[1]
            \State Initialize Arraylist $q$ 
            \Comment{Store the index of begin book of each shelf}
            \Procedure{minpower}{$i, j$}
                \If {\textit{i = n}}
                  \State \textbf{return $W-(\sum_{k = j}^i w_k)^3$}
                  \Comment{The result by definition is just $x_j$}
                \Else
                \If {$x_j < w_{i+1}$} 
                \Comment{When the book cannot fit in current shelf}
                  \State Record index $q.add(0, i+1)$
                  \State \textbf{return} $minpower(i+1, i+1) + x_{j}^3$
                \ElsIf{$minpower(i+1,j) > minpower(i+1,i+1) + x_j^3$}
                  \State Record index $q.add(0,i+1)$  
                  \Comment{Add to the first position in $q$}
                  \State \textbf{return} $minpower(i+1, i+1) + x_{j}^3$
                \Else
                  \State \textbf{return $minpower(i+1, j)$}
                \EndIf
            \EndIf
            \EndProcedure
          \end{algorithmic}
        \end{algorithm}
        \end{minipage}\\\\
    For the above algorithm, there is few things need to clarify. We define $i$ 
    as the index of the current book(the last book we already put in shelf), and
    $j$ as the index of the first book in the current shelf. The algorithm has 
    a recursive call $minpower(i,j)$ that would first determine if the next book
    $(i+1)_{th}$ book can be put in the current shelf $j$. If not, we call recursive
    call $minpower(i+1,i+1)$ putting $(i+1)_{th}$ book in the next shelf and assign
    the first book in the next shelf to be the $(i+1)_{th}$ and the waste is going to
    be add the space $W-(\sum_{k = j}^i w_k)^3$. Else, the $(i+1)_{th}$
    book can be fit in $j_{th}$ shelf and here we consider two cases if we should
    include ($(i+1)_{th}$) book in this shelf or put it in the next shelf. If don't
    include is going to minimize $x_j^3$, add the total cost of the current shelf
    with the power need if we put $(i+1)_{th}$book in the next shelf. Else
    just return $minpower(i+1,j)$, the total power used if we include the book in 
    this current level. Every time changing shelf is the better or only solution
    , we just record it in the first place in $q$. In the end it would result in 
    a beautiful list with all the indexs of books at the begining of each shelf.

    \textbf{Proof of Correctness:}\\
    The algorithm would correctly return an list of all the first indexes of each
    shelf at completion.\\
    \textbf{Claim:} The algorithm would correctly give a list of all the indexes
    of the first book in each shelf.\\
    \textbf{Base Case:} Our base case is the last call to the recursive call which
    is $minpower(i,j)$, where i is equal to n. By the definition of $i$, the last
    book is already in the shelf and in this case , the algorithm returns the waste
    of this level which is $W-(\sum_{k = j}^i w_k)^3$.\\
    \textbf{Inductive Hypothesis:} P(x) Everytime we call $minpower(a,b)$ we are going 
    to find the minimum power used in shelves from $(a+1)_th$ book to $n_th$ book.\\
    \textbf{Inductive Steps:} We need to prove that $minpower(a-1,b)$ also find 
    the minimum power used in all the shelves from $(a)_th$ book to $n$ book. 
    Since at this case, $a-1$ cannot equal to $n$, therefore no need to worry about
    line 3 to 4. Then we call $minpower(a-1,b)$, there is three possible situations.
    \begin{itemize}
        \item If $a_{th}$ book cannot fit in this current shelf, then the algorithm
        is going to call $minpower(a+1, b)$, where $b = a+1$, and by our Inductive
        Hypothesis that is going to get the minimum power used form $(a+1)_th$ 
        book to $n$ and we add the waste on this level is still the smallest waste,
        this case holds. We add the index of the $(a+1)_{th}$ book to the $q$, 
        because this is when we start a new line.
        \item If $a_{th}$ book can fit in this current shelf, and the power of 
        including this $a_{th}$ book in the current shelf is greater than the power
        making $a_th$ book the first book in the next shelf. Since both call to 
        $minpower()$ on line 9 is going to return the correct output by inductive
        hypothesis, we chose the solution that put $a_{th}$ book in the 
        next shelf. And every time we choose this possibility, we are adding the 
        index of $(a+1)_{th}$ book to the $q$, which makes sure the index is recorded.
        And we return the totoal waste used by now to the upper level by adding waste
        on this level $W-(\sum_{k = j}^i w_k)^3$ to the $minpower(a+1, b)$.
        If power of 
        including this $a_{th}$ book in the current shelf is smaller than the power
        making $a_th$ book the first book in the next shelf. We choose to put the 
        book in the current shelf and the algorithm calls $minpower(a+1, b)$ which 
        by hypothesis is going to return the minimum power used from $(a+1)_{th}$ to 
        $n_{th}$ book. So here in whichever case, we are choosing the one that uses
        smaller power, which means we are giving the minimized power from $a_{th}$
    \end{itemize}
    In any case, the algorithm is going to help us minimize the power from $a_{th}$
    book to $n_{th}$ book and since we record the first index of each shelf whenever
    we choose to put book on the new shelf. By induction, the algorithm is going to
    find the minimized power in all the shlef and since we record all the index 
    of the first book in each shelf. It will give us an list of all the index of 
    the first book in each shelf.\\
    \textbf{Time complexity:}\\ Since there is $n_th$ book to consider and the worst 
    case for y is $n$ shelves. And every time in the algorithm we are just doing 
    comparsion and addition, since the value from next level is already known. 
    Therefore, the time complexity is going to be $O(n)$.
\end{enumerate}
\end{document}

