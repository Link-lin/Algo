\documentclass[14pt]{article}
\usepackage[letterpaper, margin=1in]{geometry}
\usepackage{amssymb}
\usepackage{amsmath}
\usepackage{graphicx}
\usepackage{algorithm}
\usepackage{algorithmicx}
\usepackage{enumitem}
\usepackage[noend]{algpseudocode}
        
\def\BState{\State\hskip-\ALG@thistlm}
        
\topmargin = -50pt
        
\title{HW 7}
\author{Yunhao Lin}
        
\begin{document}
\maketitle
\begin{enumerate}
    %\item Calling RANDY(l,u) would return $(i,a_i)$ where i is a uniformly integer
    %from l to u and $a_i$ is the ID of corresponding file. Let $x$ denotes the $i$
    %value returned from RANDY(). Domain of x is between l and u. And then the expected 
    %value of x is going to be 
    
    \item Calling RANDY(l,u) would return $(i,a_i)$ where i is a uniformly integer
    from l to u and $a_i$ is the ID of corresponding file. Suppose we have a number
    for all of the file from 1 to n, 0 means it's not possible, 1 means it's visited
    and -1 means not yet determined. The initial value is -1 means not determined. 
    According to the algorithm, if the file is visited, it would be compared to x 
    to check if it's the desired file and then return. So what we need to check is
    only the possbility for some file in the list to be visited. \\
    Assume we have index $j$ and $A[j] \leq x < A[j+1]$. Then our goal is to find
    $j$.
    After we get $a_i$ and $i$ from the RANDY() call, if $x < a_i$ we mark all the 
    the files on the right of i to 0, else mark all the files on the left to be 0.
    So for each file the possiblity to be visited is $\frac{1}{i-j}$ when $i > j$ and
    $\frac{1}{j-i+1}$ when $i \leq j$  \\
    \[
    p(A[i] == 1) = \left\{
                \begin{array}{ll}
                    \frac{1}{j-i+1},   (i \leq j)\\
                    \frac{1}{i-j}, (i>j)    
                \end{array}
              \right.
    \]
    So our possiblity to find j is then $\sum_{i=1}^j \frac{1}{j-i+1} + \sum_{i=j+1}^n
    \frac{1}{i-j}$. Which is smaller than $\sum_{i=1}^n \frac{1}{i} + \sum_{i=1}^n \frac{1}{i}
    $. By the harmonic series, $\sum_{i=1}^n \frac{1}{i}$ is eqaul to O(logn).
    So our algorithm would find the correct index for x in $2O(logn) = O(logn)$
    accesses of files.  
   
    \item 
    \begin{itemize}
    \item In order to determine the expected length of any single arc. Since we
    are picking $k$ points on the circle. Then at the end we will have $k$ arcs 
    and the sum of their expectation would be 1, since this is a circle of unit 
    circumference. Suppose each arc is $X_i$, then we get the formula $\sum_{i=1}
    ^k E[X_i] = 1$. From here we can simply see that the expected value for sum 
    $X_i$ is going to be $\frac{1}{k}$. So we get $E[X] = \frac{1}{k}$. The 
    expected value of any single arc is $\frac{1}{k}$.
    \item In order to find the file with ID x that makes at most $O(\sqrt{n})$
    calls. We need to find the closest bound on both end of the file with ID $x$.\\
    \begin{minipage}{\linewidth}
            \begin{algorithm}[H]
              \caption{Find file x}
              \begin{algorithmic}[1]
                \State Initialize $small = 0$;\Comment The closest index on the left
                \State Initialize $large = inf$; \Comment The closest index on the right
                \For{i = 1 to $\sqrt{n}$}
                    \State r = RANDY();
                    \If{r == x}
                        return x;
                    \EndIf
                    \If{r $>$ small and r $<$ x}
                        small = r;
                    \EndIf
                    \If{r $<$ large and r $>$ x}
                        large = r;
                    \EndIf
                \EndFor
                \For{j = $small$ to $large$}
                    \If{NEXT(j) == x}
                        \State return x;
                    \EndIf
                \EndFor
              \end{algorithmic}
            \end{algorithm}
        \end{minipage}\\\\
    Here we first call $\sqrt{n}$ times of the RANDY(). And every time we can find
    a index either larger than $x$ or smaller than $x$. Then we are updating the bound 
    on the lower side and higher side in order to get the bound close to $x$. If
    we find $x$ from Call RANDY(), we just return right away. If not we will still
    get the bound on lower side and higher side. And
    after we get the bound on both side, we then call NEXT() from the lower bound
    to the higher bound. Since the list itself is sorted, we are sure our File with
    index $x$ is in this range. And we would be able to find it.\\
    \textbf{Time Complexity:} From part(a) we know that the expected arc length of the 
    any arc would be $\frac{1}{k}$, where k stands for the number of slices. In the
    first for loop, basically what we did is to choose random number in the circle
    of the file. And therefore the length between $small$ and $large$, the range
    our File x is, is going to be $\frac{n}{\sqrt{n}} = \sqrt{n}$. And we call the
    first loop $\sqrt{n}$ times and in each iteration all we did is compare and RANDY().
    The time complexity for the first loop would be $O(\sqrt{n})$. The second for
    loop would Call NEXT() on this range for $\sqrt{n}$ times and the time 
    complexity would be $O(\sqrt{n})$. So in total, time complexity of the algorithm
    is $O(\sqrt{n}) + O(\sqrt{n}) = 2O(\sqrt{n}) = O(\sqrt{n})$.
    \end{itemize}
    

\end{enumerate}


\end{document}