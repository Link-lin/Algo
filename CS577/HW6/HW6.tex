\documentclass[14pt]{article}
\usepackage[letterpaper, margin=1in]{geometry}
\usepackage{amssymb}
\usepackage{amsmath}
\usepackage{graphicx}
\usepackage{algorithm}
\usepackage{algorithmicx}
\usepackage{enumitem}
\usepackage[noend]{algpseudocode}
        
\def\BState{\State\hskip-\ALG@thistlm}
        
\topmargin = -50pt
        
\title{HW 6}
\author{Yunhao Lin}
        
\begin{document}
\maketitle
\begin{enumerate}
    \item In order to maximize our profit from delivering packages, first we still
    suppose $i$ is the delivery we are going to make, and $Y$ is the total time 
    used for all the previous deliveries. Sort the delivery by due date. We call our method $maxpro(i, Y)$. Every time
    we decide if the $i_{th}$ delivery should be included in the delivery schedule
    or should we exclude it. By including the $i_{th}$ delivery, we pass the next
    delivery and plus the total time used $Y$ by $t_i$. And here we add the $p_i$
    so thorough all the recursive calls, we can get the total money earned, since
    this is the case where we need to delivery the $i_{th}$ item. Also if we exclude
    the $i_{th} $ item, then we continue to the next delivery and pass $Y$ as it 
    is. From these two, we chose the maximum profit. The expression is listed below:
    \[
    maxpro(i, Y) = max\left\{
                \begin{array}{ll}
                  maxpro(i+1, Y+t_i) + p_i\\
                  maxpro(i+1, Y)
                \end{array}
              \right.
    \]
    And here is the pseudo-code for the algorithm:\\
    \begin{minipage}{\linewidth}
        \begin{algorithm}[H]
          \caption{Maximum profit}
          \begin{algorithmic}[1]
            \State Initialize Arraylist $q$ 
            \State Sort the due date by ascending order
            \Procedure{maxpro}{$i, Y$}
                \If {\textit{i = n+1}}
                  \State \textbf{return $0$;}
                  \Comment{the bundary situation}
                \Else
                \If {$Y + t_i -1 > d_i$} 
                \Comment{the item cannot be delivered}
                  \State \textbf{return} $maxpro(i+1, Y)$
                \ElsIf{$maxpro(i+1, Y+t_i) +p_i > maxpro(i+1,Y)$}
                  \State Record index $q.add(0,i+1)$  
                  \Comment{Add to the first position in $q$}
                  \State \textbf{return} $maxpro(i+1, Y+t_i) + p_i$
                \Else
                  \State \textbf{return $maxpro(i+1, Y)$}
                \EndIf
            \EndIf
            \EndProcedure 
          \end{algorithmic}
        \end{algorithm}
    \end{minipage}\\\\
    
    \textbf{Proof of Correctness:}\\
    \textbf{Claim:} The algorithm would correctly return an list of deliveries 
    to make.\\
    \textbf{Base case:} Base case is when $i = n$. When we maxpro with ($i = n$),
    there is three different situations. If $Y$, the time used before the delivery
    of the $i_{th}$ item plus the time to delive the $i_{th}$ item is longer than
    the deadline of this item, then we can only choose to not include this item and
    return maxpro(i+1, Y) which would be 0. And if we can include this last item 
    grants more profit as in line, here we record the index because that's when
    we need to deliver the package. At the last case when $i =n$, both function
    call on line eight would return 0 but HS has the $p_i$ plused. So it guarentee
    maxprofit if the item is eligible to be delivered.\\
    \textbf{Inductive Hypothesis:} 
    When $i = k$, when $k < n$, the function call would return the maximum profit 
    from $k+1$ to $n$ packages.\\
    \textbf{Inductive Steps:} For this we need to prove that when $i = k-1$, the 
    algorithm would still return the maximum profit from $k$ to $n$. When the
    algorithm calls on $maxpro(k-1, Y)$, there is three possible situations.
  \begin{itemize}
   \item If $(k-1)_{th}$ package cannot be delivered, then we can only choose not
   to deliver this $(k-1)_{th}$ package. And we call on maxpro(k, Y) and by the
   inductive hypothesis, it is going to return the maximum profit from $k+1$ to
   $n$ package. Since we are not delivering this $(k-1)_{th}$ item, the maximum 
   profit is the max porfit from $k_{th}$ package to the last pacakage.
   \item If the item can be delivered, then here we compare the maxprofit of including
   the package for delivery and not include it for delivery. As in line 8, we already
   know from the inductive hypothesis that $maxpro(k, Y+ti)$ and $maxpro(k, Y)$ is
   going to return the maximum profit from $k$ to $n$ packages. So comparing these
   would help us decide if we need to include the package, which is the $p_i$ term
   on the Left Hand Side. If yes, then record the package and return the 
   maxprofit now including the profit of this $k_{th}$ package. Else return the max profit
   not excluding the $k_{th}$ package. In any case, the profit is maximized from
   $k$ to $n$ packages after $maxpro(k-1,Y)$.
  \end{itemize}  
   So $P(k)$ holds indicates $P(k-1)$, where $k-1>0$. By induction, the algorithm
   would correctly return the max profit. And since we record the choice that would
   result in this max profit at each point, we can get a sorted list of all the 
   packages to deliver.\\
   \textbf{Time complexity:} Since there is $n$ number of packages and the worst
   case is to check $Y > T$ at the last package, the size of the subproblems would
   be $O(nT)$ and each subproblem takes $O(1)$ time to compute since the value from
   next level is already calculated and all we do is compare and addition. So in 
   total the time complexity of this algorithm would be $O(nT)$.
      
    \item  First sort the $d_i$ in descending order to an array called $\alpha$ and we declare 
    k as the indexs of $\alpha$, k is from 1 to need. So the first index of k is 
    the latest deadline which is T as defined in 1. We call our recursive function
    maxdel(k, j). Here, k is index of the sorted array of deadline and 
    j is the number of deliveries in the previous $k_th$ packages. \\
    \[
    maxdel(k, j) = min\left\{
                \begin{array}{ll}
                  maxdel(k+1, j+1) - t_i\\
                  maxdel(k+1, j) - (d_{k+1}-d_k)
                \end{array}
              \right.
    \]
    Here $t_i$ is the time need to finish the delivery at the due date 
    $\alpha[k+1]$. We are going to record all the value in an 2-D array.

    And here is the pseudo-code for the algorithm:\\
       \begin{minipage}{\linewidth}
            \begin{algorithm}[H]
              \caption{Maximum delivery}
              \begin{algorithmic}[1]
                \State Sort the pacakge as their deadline in descending order 
                \State Name this sorted array as$\alpha$
                \State Initialize a 2-D array maxdel to all -1
                maxdel[n, 0] = T;
                \For{k  = n+1 to 1}
                    \For{j = 1 to k}
                        \If{k = n+1}
                            \State maxdel[k, j] = 0;
                            \Comment{Base case}
                        \EndIf
                        \If{$min(maxdel[k+1, j+1] -t_i, maxdel[k+1,j] -(d_{k+1}-d_k)) - t_i < 0$}
                           \State maxdel[k,j] = maxdel[k+1, j];
                        \ElsIf{$ maxdel[k+1, j+1] -t_i \leq maxdel[k+1,j] -(d_{k+1}-d_k)$}  
                            \\\Comment $t_i$ is corresponding to $k$
                            \State  maxdel[k, j] = $\alpha[k]$ - $t_i$ + 1
                        \Else
                            \State maxdel[k, j] = maxdel[k+1,j]
                        \EndIf
                   \EndFor
                \EndFor\\
                \For{k = n to 1}
                    \For{j = k to 1}
                        \State find max j of maxdel[k, j] that is not -1
                        \Comment Find the maximum j value
                    \EndFor
                \EndFor
                \State Find the traverse that lead to the maximum j is the
                delivery order we want.
              \end{algorithmic}
            \end{algorithm}
        \end{minipage}\\\\\\\\
    \textbf{Proof of Correctness:}\\
    The algorithm above is going to find the maximum delivery number and the order
    to deliver.\\
    \textbf{Base case:} When we call $MinDdl(n,0)$, we reached our base case where no 
    deliveries is made. So in this case, the total time remaining for us to deliver is $T$. So, 
    we set $MinDdl[n,0]$ equal to $T$. Since we would call $MinDdl(n+1,j)$ where we actually 
    don't have $(n+1)^{th}$ item, we set $MinDdl[n+1,j]$.
    \textbf{Inductive Hypothesis:} For k = s, s is between 1 and n, maxdel(s, j) 
    is going to return the minimum deadline available, and j holds the value of the
    total deliverys in the previous s packages.\\
    \textbf{Inductive Steps:} In order to prove we need to verify that when 
    k = s-1, maxdel(s,j) would still get the correct minimum deadline.
    In the case there is coulple situations that might happen.
    \begin{itemize}
        \item First if the current package cannot be delivered in the current 
        minimum deadline, then we choose to exluce it and keep the deadline as 
        it is.
        \item If the package can be fit in the schedule, and 
        $maxdel[k+1, j+1] -t_i \leq maxdel[k+1,j] -(d_{k+1}-d_k)$
        Therefore we decide to include the
        package in the delivery then update the deadline
        to be the start state of the current package. Since in order to deliver
        the package, we need to start at least at $\alpha[s]- t_i +1$. Here, $t_i$
        is the corresponding time need to deliver the package.
        \item If we decide not to include the package in the delivery, then just
        change the minimum due date to the due date of the next item to deliver.
        Therefore, this is going to guarentee us the minimum deadline is met at 
        the end. 
    \end{itemize}
    So, by induction, the algorithm is going to return the minimum deadline. So
    at the end our deadline should be very close to 0. We save all the possiblities
    during the recursion and after the recursion finished we know all the j values
    thourgh out the possiblities and stored them in the 2d array. The algorithm
    traverse the array and find the maximum j possible and choose one path that 
    led to j. So the algorithm would determine which deliveries to make and in 
    what order to maximize the number of deliveries we make.\\
    \textbf{Time complexity: } Since the two for loop both has O(n), the time
    complexity here is  $O(n^2)$ and we need to find the maximum j and the route 
    to j is also $O(n^2)$. So the total time complexity is $O(n^2)$

\end{enumerate}


\end{document}