\documentclass{article}
\usepackage{fullpage}

\usepackage{epsfig}
\usepackage{amsfonts}
\usepackage{amssymb}
\usepackage{amstext}
\usepackage{amscd}
\usepackage{amsmath}
\usepackage{amsthm}
\usepackage{times}
\usepackage{graphicx}
\usepackage{alltt}
\usepackage{algpseudocode}

\begin{document}

%\thispagestyle{empty}

\noindent
\fbox{
\parbox{\textwidth}{
\begin{Large}
{\bf CS 577: Introduction to Algorithms\hfill Induction Review}
\end{Large}
}}

\subsection*{Solutions}

\begin{enumerate}
\item $n = 1$: We have $1 = \frac{1(1 + 1)}{2}$, so the formula trivially holds true.\\
Inductive step: Suppose that for $n = k$ that the formula holds (inductive hypothesis). Now we need to show that for $n = k + 1$, the formula still holds. With the inductive hypothesis, we have
\begin{equation*}
1 + 2 + \ldots + k = \frac{k(k + 1)}{2}.
\end{equation*}
Adding $k + 1$ to both sides, we have
\begin{equation*}
\begin{split}
1 + 2 + \ldots + k + (k + 1) & = \frac{k(k + 1)}{2} + (k + 1) \\
                             & = \frac{k(k + 1)}{2} + \frac{2(k + 1)}{2} \\
                             & = \frac{(k + 1)(k + 2)}{2},
\end{split}
\end{equation*}
thus giving us the formula for $n = k + 1$.

\item $n = 1$: We have $1^2 = \frac{2(1^3) + 3(1^2) + 1}{6}$, so the formula trivially holds true.\\
Inductive step: With the inductive hypothesis, we have
\begin{equation*}
1^2 + 2^2 + \ldots + k^2 = \frac{2k^3 + 3k^2 + k}{6}.
\end{equation*}
Adding $(k + 1)^2$ to both sides, we have
\begin{equation*}
\begin{split}
1^2 + 2^2 + \ldots + k^2 + (k + 1)^2 & = \frac{2k^3 + 3k^2 + k}{6} + (k + 1)^2 \\
                                     & = \frac{2k^3 + 3k^2 + k}{6} + \frac{6(k^2 + 2k + 1)}{6} \\
                                     & = \frac{2k^3 + 9k^2 + 13k + 6}{6} \\
                                     & = \frac{2(k + 1)^3 + 3(k + 1)^2 + (k + 1)}{6},
\end{split}
\end{equation*}
thus giving us the formula for $n = k + 1$.

\item $n = 0$: We have $S = \emptyset$, so the only possible subset of $S$ is $\emptyset$; $2^0 = 1$, thus the claim holds.\\
Inductive step: In the inductive hypothesis, we suppose that for any set $S$ that has $k$ elements, the number of subsets of $S$ is $2^k$. Now if we have a set $S'$ of size $k + 1$, we can derive a set $S$ of size $k$ by removing some element $\alpha \in S'$, i.e., $S = S' \setminus \alpha$. By our hypothesis, there are $2^k$ subsets of $S$. Any subset of $S'$ is either a subset of $S$ or can be derived by adding $\alpha$ to some subset of $S$. Therefore, the number of subsets of $S'$ is $2(2^k) = 2^{k + 1}$, and so the claim holds for $n = k + 1$.

\item $n = 0$: We have ${0 \choose 0} = 2^0 = 1$, so the claim holds.\\
Inductive step: In the inductive hypothesis, we suppose
\begin{equation*}
{k \choose 0} + {k \choose 1} + \ldots + {k \choose k - 1} + {k \choose k} = 2^k.
\end{equation*}
Duplicating the terms in the left-hand side and multiplying the right-hand side by 2 yields the following:
\begin{equation*}
{k \choose 0} + {k \choose 0} + {k \choose 1} + {k \choose 1} + \ldots + {k \choose k - 1} + {k \choose k - 1} + {k \choose k} + {k \choose k} = 2^{k + 1}.
\end{equation*}
Acknowledging that ${k \choose 0} = {k + 1 \choose 0} = 1$ and ${k \choose k} = {k + 1 \choose k + 1} = 1$, we get
\begin{equation*}
{k + 1 \choose 0} + {k \choose 0} + {k \choose 1} + {k \choose 1} + {k \choose 2} + \ldots + {k \choose k - 2} + {k \choose k - 1} + {k \choose k - 1} + {k \choose k} + {k + 1 \choose k + 1} = 2^{k + 1}.
\end{equation*}
Using the hint, condensing the inner $2k$ terms on the left-hand side in the above equation yields
\begin{equation*}
{k + 1 \choose 0} + {k + 1 \choose 1} + {k + 1 \choose 2} + \ldots + {k + 1 \choose k - 1} + {k + 1 \choose k} + {k + 1 \choose k + 1} = 2^{k + 1},
\end{equation*}
thus giving us the formula for $n = k + 1$.\\
\textit{Note:} Notice the connection between this problem and the last. The formula in this problem is an alternative representation of the claim in the last problem.

\item We first prove that $x_n < 2$ for all $n$.\\\\
$n = 1$: $\sqrt{2} < 2$, thus the claim trivially holds.\\
Inductive step: In the inductive hypothesis, we suppose $x_k < 2$. Thus, $x_k + 2 < 2 + 2 = 4$, and furthermore, $\sqrt{x_k + 2} < \sqrt{4}$, or equivalently, $x_{k + 1} < 2$, thus proving our claim for $n = k + 1$.\\\\
We next prove that $x_n < x_{n+1}$ for all $n$.\\\\
$n = 1$: $x_1 = \sqrt{2}$ and $x_2 = \sqrt{2 + x_1} = \sqrt{2 + \sqrt{2}}$, thus $x_1 < x_2$ trivially holds.\\
Inductive step: In the inductive hypothesis, we suppose $x_k < x_{k + 1}$. By adding 2 and taking the square root of both sides of the inequality, we arrive at $\sqrt{x_k + 2} < \sqrt{x_{k + 1} + 2}$, or equivalently, $x_{k + 1} < x_{k + 2}$, thus proving our claim for $n = k + 1$.

\item We will use induction to prove an alternative claim that the grid can be tiled such that all but one square is covered, with the hole being anywhere we choose.\\\\
$n = 0$: We have a 1-by-1 grid, i.e., a unit square. Without laying any tiles, we already achieve the only possible tiling that leaves one square exposed, so the claim trivially holds.\\
Inductive step: In the inductive hypothesis, we suppose that a $2^k$-by-$2^k$ grid can be tiled such that the hole can be anywhere. To similarly tile a $2^{k + 1}$-by$2^{k + 1}$ grid, we can break up the grid into four smaller $2^k$-by-$2^k$ grids. Upon determining which quadrant we would like to place the hole, we tile the other three quadrants such that for each of those quadrants, all tiles except for the corner square lying in the center of the larger $2^{k + 1}$-by$2^{k + 1}$ grid is covered. We then place a tile to cover those three uncovered squares, leaving us with an uncovered $2^k$-by-$2^k$ quadrant. We tile this last quadrant such that the hole is where we would like it in the larger $2^{k + 1}$-by$2^{k + 1}$ grid, thus proving our claim for $n = k + 1$.\\\\
This general claim suits our needs because it states that a tiling exists regardless of where the hole is located. Therefore, a tiling such that all but one of the four center squares is covered will exist.

\item $n = 0$: $0! = 1$, and the program returns 1 accordingly.\\
Inductive step: In the inductive hypothesis, we suppose that the program returns $k!$ when given $k$ as input. When the program is given $k + 1$ as input, it correctly returns $(k + 1) \cdot Factorial(k) = (k + 1)!$, thus proving the correctness of the program.

\item Recall how strong induction differs from weak induction. All of the previous problems could be solved using weak induction, which means that given a base case exists, only the $k$th case was needed to show that the $(k + 1)$th case holds. Strong induction allows us to say that the 1st, 2nd, $\ldots$ , $(k - 1)$th, and $k$th cases can all be used to arrive at the $(k + 1)$th case.\\\\
$n = 0$: When the list is empty, the algorithm simply returns -1. Thus, the algorithm behaves correctly.\\
Inductive step: In the inductive hypothesis, we suppose that the algorithm behaves correctly for all lists of sizes 0 through $k$ inclusive. For a list of size $k + 1$, if the median (middle-left element if $k + 1$ is even) matches $x$, then the algorithm returns its index. If $x$ is less than the median, then $x$, if it exists in the list, must be present in the sublist formed by taking all elements to the left of $x$ in the original list; note that the length of this sublist is between 0 through $k$ inclusive. Thus, the algorithm behaves correctly by
recursing on this sublist. A similar argument follows when $x$ is greater than the median. Therefore, the overall algorithm is correct.

\end{enumerate}

\end{document}